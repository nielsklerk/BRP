\documentclass[twoside, single, authoryear, semicolon, 12pt]{lion-msc}
\usepackage{lipsum}
\usepackage{booktabs}
\usepackage{pifont}
\usepackage[version=4]{mhchem}
\usepackage{parskip}
\usepackage{float}
\usepackage{makecell}
\usepackage{graphicx}
\usepackage{subcaption}
\usepackage{caption} 
\usepackage{amsmath}

\title{Detecting Nitrogen Carriers in Planet-Forming Regions of Protoplanetary Disks}
\author{Niels de Klerk}

\major{Astronomy and Physics}
\affiliation{Leiden Observatory, Universiteit Leiden}

\newdate{date}{\day}{\month}{\year}           % definition of time and date using datetime package
% \newdate{date}{27}{08}{2010}
\date{\displaydate{date}}

\studentid{s3640477}                           % check you student ID, LaTeX does not do this
\abstract{Nitrogen is one of the most important elements for life on earth, and plays an important role in the chemistry of protoplanetary disks that are the birthplace for planets. However, the only nitrogen carrier detected in these disks is HCN. This thesis aims to investigate the detection of NO and NH\3 in the spectra taken by JWST MIRI MRS. For this, a grid of ProDiMo models with varying C and O abundances was used in combination with FLiTs to create spectra. The effects of the C and O abundances on the flux were investigated, and spectral regions of interest were selected. Furthermore, a new technique of detecting molecules using cross-correlation was studied. Utilizing this technique, molecules already detected in GWLup, Sz98, and V1094Sco were reaffirmed with a possible detection of NO in V1094Sco.}% limit your self to 1/2 page or 500 words
\dailysupervisor{Msc. Marissa Vlasblom \\ \hspace*{\fill}Msc. Aditya M. Arabhavi}
\supervisor{Prof. Dr. Ewine van Dishoeck \\ \hspace*{\fill}Prof. Dr. Inga Kamp} % Note that this should be a LION staff member!
\corrector{Dr. Matthieu Schaller}                      % This could be a LION staff member or your external supervisor

\degree{Bachelor of Science}                     % The default option is "Bachelor of Science", change if needed

\major{Astronomy and Physics}                  % The default option is "Physics", change if needed
%\major{Physics and Mathematics}

% optional cover picture - should be jpg or pdf
% \coverpicture{\includegraphics[width=2cm]{Latex/lion-msc-logo.pdf}}

% Use this to make hyperlinks visible in the document.
% \hypersetup{colorlinks=true}

% ---------------------------------------------------------------- My defintions!
\renewcommand{\vec}[1] {\ensuremath{ \overrightarrow{ #1 } }}
% \renewcommand{\vec}[1] {\ensuremath{ \mathbf{ #1 } }}
% \bra \ket \braket and \proj
\newcommand{\bra}[1]{\ensuremath{\langle #1 \vert}}
\newcommand{\ket}[1]{\ensuremath{\vert #1 \rangle}}
\newcommand{\braket}[2]{\ensuremath{\langle #1 \vert #2 \rangle}}
\newcommand{\proj}[1]{\ensuremath{\vert #1 \rangle \langle #1 \vert}}

\newcommand{\kpar}{\ensuremath{k_\parallel}}

\newcommand{\4}{$_4$}
\newcommand{\3}{$_3$}
\newcommand{\2}{$_2$}
% ----------------------------------------------------------------

% \usepackage{tocloft}
% \renewcommand{\cftchapdotsep}{\cftdotsep}
\begin{document}

% roman numbering in the table of contents section
\pagenumbering{roman}

\maketitle

% Table of contents:  it is a good idea to include this into your thesis
\tableofcontents
\cleardoublepage
\pagenumbering{arabic}
\chapter{Introduction}
The nebular hypothesis was first proposed in \textit{The Principia} by Emanuel Swedenborg in 1745. Immanuel Kant developed the theory further in 1752 and later modified by Pierre-Simon Laplace in 1796. The theory states that a planetary system is formed from a slowly rotating gas cloud that collapses into a disk. Centuries later, the first protoplanetary disk was observed by O'Dell using the Hubble Space Telescope \citep{ODell1993}. In recent years, the Atacama Large Millimeter/submillimeter Array (ALMA) has imaged a large collection of these protoplanetary disks, showing a wide variety of structures and compositions (e.g. \cite{gardner2025exoalmaxialmaobservations, shoshi2025alma2dsuperresolutionimaging}). The JWST MIRI mid-INfrared Disk Survey (MINDS) team uses the JWST to investigate the inner parts of protoplanetary disks (e.g. \cite{Arabhavi_2025, Vlasblom_2025}). The study of protoplanetary disks is important to find answers to the fundamental questions, 'How did life arise?' and 'Are we alone in the universe?'.

The spectrum of protoplanetary disks can tell us a lot about the properties of the disk and the host star. An important example is an observation of the disk around a T Tauri star Sz98 \citep{Gasman_2023}. Using MIRI on the JWST, they probed the inner regions of the disk. They detected CO$_2$, H$_2$O, OH, CO, and HCN. Furthermore, no other organics were detected, suggesting a low C/O ($<$0.5) ratio. This result differed from the ratio found using ALMA ($>$1), which probed the outer regions. This highlights the complexity of disks and their chemistry.

However, this observation is not necessarily representative of disks. \cite{colmenares2024jwstmiridetectioncarbonrichchemistry} observed a disk around the T Tauri star DoAr 33 with an exceptionally high C/O ratio of 2-4. In addition to CO, H\2O, and CO\2, like in the Sz98 disk, the more complex carbohydrates C\2H\2 and C\4H\2 were found. The presence of these molecules is indicative of this high ratio. A possible explanation for this carbon-rich environment is the slow accretion rate of the star, which results in slowing the radial mixing and the persistence of the carbon-grain destruction.

Observations of the protoplanetary disk around the T Tauri star GW Lup have given the first detection of $^{13}$CO$_2$ in a protoplanetary disk. \cite{Grant_2023} The combination with the spectral resolution of the JWST-MIRI and the high SNR allows for the detection of weaker spectral features. Notably, the deduced N$_{CO_2}$/N$_{H_2O}$ was significantly higher than previously thought. This could indicate a cavity between the H$_2$O and CO$_2$ snowlines. These findings demonstrate the new possibilities that JWST offers for studying disk structures. 

% The effects of a cavity on the spectrum were studied by \cite{vlasblom2023midinfraredspectrattauri}. It was speculated that 'CO$_2$-only sources' have inner cavities which extend beyond the H$_2$O snowline, but stay within the CO$_2$ snowline. By running thermo-chemical models with different inner cavity sizes, it was shown that N$_{CO_2}$/N$_{H_2O}$ grows as the size of the cavity grows and then sharply drops. The conclusions drawn from this can help to interpret the spectra of disks.

\chapter{Theoretical Background}
\section{Formation and Evolution of Protoplanetary Disks}
\subsection{Formation of Disks}
Molecular clouds are large collections of gas and dust. When these clouds are perturbed, parts of the cloud can become dense enough, making them collapse under their own gravity. This will result in the creation of a star surrounded by gas and dust. The angular momentum of each of the particles is randomly oriented, but on average, the angular momentum vector is pointed in a certain direction. Two particles colliding will cause them to exchange angular momentum. If many collisions happen, all the angular momenta other than the average angular momentum vector get canceled out. This results in the formation of a protoplanetary disk around the pre-main-sequence star. Around the protostar and the disk is the approximately spherically shaped envelope of material that did not get distorted due to rotation. 

The evolution of the disk can be categorized into 4 classes \citep{1987ApJ...312..788A}. Class 0 objects are objects where an optically thick envelope surrounds the protostar. Class I objects have a star and disk but are still embedded in an optically thick envelope. The star and disk become visible in class II objects as the star has become bright enough to blow away the envelope. In class III objects, the disk has largely been depleted of gas and dust, leaving behind planets and planetesimals. 

\subsection{Disk Dynamics}
Dust particles move around in the protoplanetary disk. Due to their relative motion with the gas, they experience drag, slowing them down. This results in the dust particles settling towards the midplane of the disk. 

Magnetic fields can act like springs between particles. When particles in adjacent orbits come close together, they can magnetically interact. These particles have different velocities, where the particle in the inner orbit has a larger speed than the particle in the outer orbit. The interaction between the particles will make their velocities more similar. As a result, the particle in the inner orbit loses angular momentum, and the particle in the outer orbit gains it. Therefore, the inner particle moves closer to the star and falls into an orbit with a higher speed, whereas the other particle does the opposite. Hence, the difference in velocity is increased, which means there is an instability. This instability is known as magneto-rotational instability (MRI), and was first described by \cite{RevModPhys.70.1}. 

When protoplanetary disks get massive, they can collapse under their own gravity. This effect is counteracted by pressure gradients and shear motion. All of these influences on the stability of the disk can be captured by the Toomre Q parameter.
\begin{equation}
    Q = \frac{c_s\Omega}{\pi G\Sigma}
\end{equation}
where the disk is stable for $Q>1$.

\subsection{Disk Dispersion}
Over time, a protoplanetary disk will evaporate due to material falling into the star or planets, or material gets irradiated by stellar and interstellar UV and X-rays. 


\subsection{Planets and Substructures}
To form planets around a star, dust particles need to clump together. However, this process is lengthy as the size of the dust particles is on the order of micrometers, and planets on the order of thousands of kilometers. Through collisions, particles can stick together to form bigger particles. However, there are several limiting factors that reduce this growth. As particles get bigger, they experience more drag, so they slow down greatly. This limits the number of collisions as the particle encounters fewer other particles. Another limitation is that when bigger particles collide, the collision can be so powerful that it breaks them up into tiny pieces. 
\textbf{more}

Gaps in the disk can form due to several reasons, but one interesting reason is the interaction with a planet.


\textbf{STRUCTURE}
The dust and gas are the two components of protoplanetary disks and evolve separately. 




2,4

% % Factors that influence the evolution



% The protoplanetary disk has structure, namely the radial structure and the vertical structure. The innermost regions ($<$10 AU) of the disk are the hottest and receive the most radiation. Most of the molecules are in gaseous form in this region, except for some silicates and metals. The star heavily influences the chemistry. The intermediate regions (10-100 AU) have lower temperatures, allowing molecules like water to condense. The outer regions ($>$100 AU) are the coldest in the disk. Molecules have formed ice here, making them inert. \\
% The height of the disk is also an important factor in its structure. The midplane is the horizontal plane through the center of the disk. Most dust particles settle around the midplane, whereas the gas particles are further out. The midplane is also colder than the outside of the disk as it receives less light from the star. The protoplanetary disk tends to flare outwards. Close to the star, the disk is fairly compact, but when you go outwards from the star, the disk gets more and more extended from the midplane. 

% \begin{figure}[H]
%     \centering
%     \includegraphics[width=\linewidth]{Figures/disk-sketch.png}
%     \caption{A schematic representation of a protoplanetary disk. The planets in the solar system are shown as a point of reference. \cite{inproceedings}}
%     \label{fig: enter-label}
% \end{figure}

\section{Disk Chemistry}
Explaining something about general chemistry, including in-depth nitrogen chemistry

\subsection{Heating and Cooling}
PAH

discuss \cite{van_Gelder_2024}
\section{Emission Mechanisms}
It is important to understand how molecules emit photons, as this can help to understand how molecules are detected in spectra. In this section, we will explain how these processes work. This is largely based on Chapter 11 of \cite{1979rpa..book.....R}. 

When a molecule transitions from a higher energy state to a lower energy state, it emits a photon with a wavelength given by

\begin{equation}
    \lambda=\frac{hc}{\Delta E}
\end{equation}

where $\lambda$ is the wavelength of the light, $h$ the Planck constant, $c$ the speed of light, and $\Delta E$ is the difference in energy between the two states. Molecules order themselves in different energy states following the rules of thermodynamics. In thermal equilibrium, they organize according to the Boltzmann distribution (\autoref{eq: boltzmann}). 

\begin{equation}
    \frac{N_i}{N}=\frac{g_i\exp{(-E_i/k_BT)}}{Z}
    \label{eq: boltzmann}
\end{equation}

with $N_i$ the number of molecules in state $i$, $E_i$ the energy of state $i$, $g_i$ the degeneracy of that energy state, $T$ the temperature of the molecule and $Z$ the partition function $Z=\sum_j g_j\exp{(-E_j/k_BT)}$.
Different transitions can take place within molecules. This section will discuss the 3 most fundamental transitions: electronic, vibrational, and rotational.

\subsection{Electronic Transition}
The transition of the highest energy is the electronic transition. In this transition, an electron jumps from an orbital with higher energy to one with lower energy. This reduces the total energy of the molecule. For simplicity, we will examine the hydrogen atom. The energy levels in the hydrogen atom are determined by: 

\begin{equation}
    E^e=-\frac{13.6\mathrm{ eV}}{n^2},
    \label{eq: electronic}
\end{equation}

where n is the principal quantum number. The energy 13.6 eV in \autoref{eq: electronic} is called the ionization energy. This is the energy an electron in the ground state needs to gain to get unbound, resulting in the ionization of the atom. The transitions to $n=1$ are called the Lyman series (Ly$\alpha$, Ly$\beta$, etc.). The Balmer series corresponds with the transitions to $n=2$ (H$\alpha$, H$\beta$, etc) and the Paschen series with $n=3$. The degeneracy of each state is $g(n)=2n^2$. The energy levels and the line strengths corresponding to transitions between these energy levels are shown in \autoref{fig: elec}.

\begin{figure}[H]
    \centering
    \includegraphics[width=\linewidth]{Figures/ElecSpectrum.pdf}
    \caption{The electronic energy levels and the corresponding spectrum, where the molecules follow the Boltzmann distribution}
    \label{fig: elec}
\end{figure}

In \autoref{fig: elec}, the collection of lines on the high-energy side corresponds with the Lyman series. The collection of lines to the left of it corresponds to the Balmer series, etc.
The electronic transitions typically occur in the ultraviolet (UV) and visible (VIS) parts of the electromagnetic spectrum. For example, the transition from n=2 to n=1, the Ly$\alpha$ transition, occurs at 121.567 nm.


\subsection{Vibrational Transition}
Vibrational transitions are corrections to the energy states set by the electronic transitions. This transition changes the vibration of the molecule. 
The bonds between atoms in a molecule can be viewed as springs connecting the atoms. These springs can have different motions, such as stretching and bending. The energies of these motions are discrete and given by: 

\begin{equation}
    E^{vib}=\hbar\omega(v+1/2)
\end{equation}

with $\hbar$ the reduced Planck constant ($\hbar=\frac{h}{2\pi}$), $\omega=\sqrt{\frac{k}{\mu}}$ with $k$ the force constant and $\mu$ reduced mass, and $v$ the vibrational quantum number. Transitions between energy states follow the selection rule: $\Delta v=\pm 1$. This means the only permitted transitions occur between energy levels with a vibrational quantum number that is one lower or higher than the current state. This results in the difference in energy between states to be:

\begin{equation}
    \Delta E^{vib}=\hbar\omega.
\end{equation}

The energy levels and the resulting spectrum are shown in \autoref{fig: vib}.

\begin{figure}[H]
    \centering
    \includegraphics[width=\linewidth]{Figures/VibSpectrum.pdf}
    \caption{The vibrational energy levels and the corresponding spectrum}
    \label{fig: vib}
\end{figure}

The vibrational transitions typically occur in the infrared (IR) part of the electromagnetic spectrum. 


\subsection{Rotational Transition}
The transition with the lowest energy is the rotational transition. This transition changes the angular momentum of the molecule. Molecules rotate with discrete energies given by:

\begin{equation}
    E^{rot}=B_eJ(J+1)
\end{equation}

where $B_e$ is the rotational constant, and $J$ is the rotational quantum number. 
The degeneracy of each state is $g(n)=2J+1$. Transitions between energy states follow the selection rule: $\Delta J=\pm 1$. This means the only permitted transitions occur between energy levels with a rotational quantum number that is one lower or higher than the current state. This results in the energy difference between a state with rotational quantum number $J+1$ to the state with rotational quantum number $J$ to be:

\begin{equation}
    \Delta  E^{rot}=2B_e(J+1)
\end{equation}

The energy levels and the resulting spectrum following the Boltzmann distribution are shown in \autoref{fig: ro}.

\begin{figure}[H]
    \centering
    \includegraphics[width=\linewidth]{Figures/RoSpectrum.pdf}
    \caption{The rotational energy levels and the corresponding spectrum, where the molecules follow the Boltzmann distribution}
    \label{fig: ro}
\end{figure}

The electronic transitions typically occur in the microwave and radio parts of the electromagnetic spectrum.

\subsection{Ro-vibrational Transition}
In nature, pure vibrational transitions are rare. More often than not, the vibrational transitions are accompanied by rotational transitions. The combination of these two transitions is called a ro-vibrational transition. Transitions between energy states follow the selection rules: $\Delta v=\pm 1$, $\Delta J=0, \pm 1$. Plotting the energy levels and the resulting spectrum following the Boltzmann distribution gives \autoref{fig: rovib}.

\begin{figure}[H]
    \centering
    \includegraphics[width=\linewidth]{Figures/RoVibSpectrum.pdf}
    \caption{The ro-vibrational energy levels and the corresponding spectrum, where the molecules follow the Boltzmann distribution}
    \label{fig: rovib}
\end{figure}

In \autoref{fig: rovib}, a distinctive shape is visible: P, Q, and R branches. The P branch is the set of peaks with a lower energy than the central peak. The Q branch is the central peak, and the set of peaks on the higher energy side of the peak is the R branch. The P, Q, and R branches correspond to $\Delta J=1, 0, -1$ respectively. For some molecules and some vibrational modes, the transition with $\Delta J=0$ is forbidden. This also means that the Q branch is missing in the spectrum.

\section{JWST MIRI MRS}
The James Webb Space Telescope (JWST) was launched on the 25th of December 2021. This new telescope was a big leap forward for space research. One of the instruments on board is the Mid-Infrared Instrument (MIRI). This instrument is used for observations in the mid-infrared, as the name suggests. MIRI has 4 observation modes in the wavelength range between 4.9 and 27.9 $\mu$m:  imaging, coronagraphic imaging, low-resolution spectroscopy, and medium-resolution spectroscopy \citep{}. 

\section{Modeling of Protoplanetary Disks}
Some of the properties of the protoplanetary disks can be directly inferred from disk measurements. For instances where this is not the case, we would still like to learn more from our data via a different method. We can use models to simulate the protoplanetary disk and generate synthetic data to compare to actual observations. There are various types of models. Ranging from 0D slab models to thermo-chemical disk models. 
\textbf{EXPLAIN WHAT FIDUCIAL IS}


\subsection{ProDiMo}
For our research, we run simulations using the Protoplanetary Disk Model (ProDiMo), a thermo-chemical disk model. ProDiMo models both the gas and dust in a disk using a 2D slice of the disk and assuming that the disk is axisymmetric. 

\begin{figure}[H]
    \centering
    \includegraphics[width=0.5\linewidth]{code structure PRODIMO}
    \caption{Caption}
    \label{fig:enter-label}
\end{figure}

Define the gas and dust densities for different radii and heights above the midplane. Set the stellar parameters: stellar mass, effective temperature, stellar luminosity, UV, and X-ray. Define interstellar properties: IR background, UV background, and Cosmic rays.  Then, it performs the radiative transfer, which yields the dust temperature and the radiation field at every point in the disk. This is followed by a chemical thermal balance, which provides the gas temperature, the density of different molecules, and the excitation levels of the molecules. 

The gas density can be found using \autoref{eq: density}
\begin{equation}
    \rho_g(r,z)=\frac{\Sigma(r)}{\sqrt{2\pi}h(r)}\exp{\left(-\frac{z^2}{2h(r)^2}\right)},
    \label{eq: density}
\end{equation}
where $\Sigma$ is the surface density, $h(r)$ the scale height (which can be calculated using hydrostatic equilibrium. There are several methods to calculate the dust density: fixed gas-to-dust mass ratio, radius-dependent gas-to-dust mass ratio, grain size-dependent gas-to-dust mass ratio, and settling. 

To find the radiation field, \autoref{eq: radiative transfer} is used.
\begin{equation}
    \frac{1}{\rho(\vec{r})}\frac{\partial I_\nu(\rho(\vec{r}, \hat{k}))}{\partial s}=-\kappa^{ext}_\nu I_\nu(\rho(\vec{r}, \hat{k})) + \kappa^{abs}_\nu B_\nu(T(\vec{r})) + \kappa^{sca}_\nu J_\nu(\vec{r})
    \label{eq: radiative transfer}
\end{equation}
where 

It is further assumed that the disk is in radiative equilibrium 
\begin{equation}
    \int^\infty_0\kappa^{abs}_\nu B_\nu(T(\vec{r}))d\nu=\int^\infty_0\kappa^{abs}_\nu J_\nu(\vec{r})d\nu
\end{equation}
which gives the dust temperature. This is iteratively repeated until the dust temperature converges.

\textbf{chemistry}

\subsection{FLiTs}
The output of ProDiMo models can be put through the Fast Line Tracing system (FLiTs)\footnote{To get access to FLiTs, contact M. Min: M.Min@sron.nl} to get an accurate spectrum of the disk. \cite{Woitke_2018} describes the development of FLiTs. FLiTs was developed as a tool to overcome some challenges that simulating spectra in the IR poses. This spectral region contains a great number of emission lines and energy levels. Disks are complex; light can be emitted in one region of the disk and reabsorbed elsewhere. Lastly, the full radiative transfer equation must be solved, as high optical depths are involved. At the time of the development of FLiTs, there was another code available named RadLite \citep{Pontoppidan_2009}. However, this code was slow and 
calculated the spectrum on a line-to-line basis, which prevented it from accurately describing line blending. So a new code, FLiTs, was developed to tackle these problems. 

All quantities, such as density and temperature, are provided from ProDiMo to FLiTs and transformed from a points-based grid to a cell-based grid. The model assumes Keplerian rotation and ignores the small effects from the pressure gradient, which can slow down the rotation of the disk by a few percent. Parallel rays are then cast through the disk at the inclination angle. The solution of the radiative transfer equation per wavelength is then integrated along these rays to produce the spectrum. To avoid aliasing, it uses randomized spatial and spectral sampling. As each part of the disk contributes to different parts of the spectrum, bundles of rays are used at random locations. The accuracy is determined by the number of rays, which means higher spatial resolution.

Compared to the single line spectrum produced by ProDiMo, FLiTs reproduced it accurately. In conclusion, FLiTs is a fast and accurate ray-tracer that can help to provide fits to observations and learn more from simulations.

\section{Research Objectives and Scope}
First, we want to investigate how nitrogen carriers depend on the abundances of carbon and oxygen. Next, we want to develop a method to detect nitric oxide (NO) and ammonia (NH\3) in the spectra of protoplanetary disks using JWST MIRI MRS. Lastly, we want to apply our methods to observations and determine upper limits on the column densities of NO and NH\3 for these sources. 
We present the methodology in \autoref{Ch: Methods}. The data and results are shown in \autoref{Ch: Results}. We discuss the results and their implications in \autoref{Ch: Discussion}. In \autoref{Ch: Conclusion}, we conclude our thesis.


\chapter{Methods}\label{Ch: Methods}

\section{ProDiMo Simulations}
To investigate whether we can detect NO and NH\3 in the spectra of protoplanetary disks and how nitrogen carriers depend on the C and O abundance, we used a grid of ProDiMo models with varying abundances of C and O. The parameters with which the models were run are given in \autoref{tab: parameters}.

\begin{equation}
    \epsilon_X\equiv\log\frac{N_X}{N_H}+12
\label{eq: abundance}
\end{equation}

The abundances in the models are defined as in \autoref{eq: abundance}, where the abundance of some element X ($\epsilon_X$) depends on the natural log of the ratio of the number density of that element ($N_X$) to the number density of hydrogen ($N_H$). In this system, $\epsilon_H=12$ by definition. The difference in abundances of C and O compared to solar abundances was varied between -0.5 and 0.5 in steps of 0.25, where a difference of 0 corresponds to the solar abundance. The grid contains 25 models, with the fiducial model in the middle, which is based on the solar abundances of C and O (see \autoref{tab: abundances}). Additionally, the nitrogen abundance was increased by one order of magnitude compared to the solar abundance to see what the effects on the spectrum are and to possibly enhance the spectral features of nitrogen carriers, thereby making their detection easier.

\begin{table}[H]
\centering
\begin{tabular}{@{}lll@{}}
                                  &                             &                            \\ \hline\midrule
\textbf{Property}                 & \textbf{Symbol}             & \textbf{Value}             \\ \midrule
Stellar mass                      & M$_\ast$                    & 0.7M$_\odot$               \\
Effective Temperature             & T$_\ast$                    & 4000 K                     \\
Stellar Luminosity                & L$_\ast$                    & 1 L$_\odot$                \\
UV excess                         & f$_{UV}$                    & 0.01                       \\
UV powerlaw index                 & p$_{UV}$                    & 1.3                        \\
X-ray luminosity                  & L$_x$                       & 10$^{30}$ erg/s              \\
X-ray emission temperature        & T$_x$                       & 2$\times10^7$ K            \\ \midrule
Strength of interstellar UV       & $\chi^{ISM}$                & 1                          \\
Strength of interstellar IR       & $\chi^{ISM}_{IR}$           & 0                          \\
Cosmic ray H$_2$ ionization rate  & $\zeta_{CR}$                & $1.7\times10^{-17} s^{-1}$ \\ \midrule
Inner disk radius                 & R$_{in}$                    & 0.05 au                    \\
Outer disk radius                 & R$_{tap}$                   & 30 au                      \\
Column density power index        & $\epsilon$                  & 1                          \\
Reference scale height            & H$_g$ (100 au)              & 10 au                      \\
Flaring power index               & $\beta$                     & 1.15                       \\ \midrule
Minimum dust particle radius      & a$_{min}$                   & 0.05 $\mu$m                \\
Maximum dust particle radius      & a$_{max}$                   & 3000 $\mu$m                \\
Dust size dist. power index       & a$_{pow}$                   & 3.5                        \\
Turbulent mixing parameter        & $\alpha_{settle}$           & 0.001                      \\
Refractory dust composition       & Mg$_{0.7}$Fe$_{0.3}$SiO\3 & 60 \%                      \\
                                  & amorph. C                   & 15 \%                      \\
                                  & porosity                    & 25 \%                      \\
PAH abundance rel. to ISM         & f$_{PAH}$                   & 0.01                       \\
Chemical heating efficiency       & $\gamma^{chem}$             & 0.2                        \\ \midrule
Distance to the observer          & d                           & 140 pc                     \\ \bottomrule
\end{tabular}
\caption{Disk parameters used for the grid of ProDiMo models}
\label{tab: parameters}
\end{table}


The output of the ProDiMo model grid was used to generate the mid-infrared spectra using FLiTs. The full spectrum includes emission from C\2H\2, CH\4, CO, CO\2, H\2O, HCN, NH\3, NO, OH. In addition, FLiTs produced spectra containing the emissions of the individual molecules. The FLiTs spectra were calculated at a very high spectral resolution ($R=1\cdot10^5$). To make the spectra more comparable to the JWST MIRI MRS spectra, the spectrum was convolved to a resolving power of $R = \lambda/\Delta\lambda = 3000$ using a Gaussian kernel to create a spectrum of the correct resolving power. The value of 3000 was chosen as it is approximately the average resolving power of the JWST MIRI MRS \citep{Argyriou_2023}. 

\begin{table}[H]
\centering
\begin{tabular}{@{}ccc@{}}
                                  &                             &                            \\ \hline\midrule
\textbf{Element} & \textbf{+12 Abundance} & \textbf{Variation}            \\ \midrule
H                & 12                     & Fixed                         \\
He               & 10.98                 & Fixed                         \\
C                & 8.14                  & {[}-0.5, -0.25, 0, +0.25, +0.5{]} \\
N                & 8.90                  & Fixed                         \\
O                & 8.48                  & {[}-0.5, -0.25, 0, +0.25, +0.5{]} \\
Ne               & 7.95                  & Fixed                         \\
Na               & 3.36                  & Fixed                         \\
Mg               & 4.03                  & Fixed                         \\
Si               & 4.24                  & Fixed                         \\
S                & 5.27                  & Fixed                         \\
Ar               & 6.08                  & Fixed                         \\
Fe               & 3.24                  & Fixed                         \\
PAH              & 3.44                  & Fixed                         \\ \bottomrule
\end{tabular}
\caption{The elemental abundances used in the simulation of the grid of ProDiMo models. The solar abundances were used, except for N, which was enhanced by one order of magnitude.}
\label{tab: abundances}
\end{table}

Noise was added to the simulated spectra to make them as realistic as possible. The observations obtained by the MINDS JWST GTO program have a signal-to-noise ratio (SNR) of 300 \citep{henning2024mindsjwstmirimidinfrared}. To make the simulated data align with an $SNR = 300$, the minimum of the simulated spectrum before continuum subtraction was used to find the variance $\sigma^2$ by using \autoref{eq: SNR}.

\begin{equation}
    SNR = \frac{F_{min}}{\sigma}\Rightarrow\sigma=\frac{F_{min}}{SNR}
    \label{eq: SNR}
\end{equation}

The variance $\sigma^2$ was then used to add Gaussian noise to each flux value following \autoref{eq: noise}.

\begin{equation}
    \tilde{F}_i = F_i + \varepsilon_i,\;\varepsilon_i\sim\mathcal{N}(0, \sigma^2)
    \label{eq: noise}
\end{equation}

After applying this noise, we subtracted the continuum. This continuum-subtracted spectrum was subsequently used for analysis.


The flux density provided by FLiTs is in units of Jansky. To obtain the flux, we integrated the flux density by first converting the flux density from per unit frequency to per unit wavelength using \autoref{eq: conversion}.

\begin{equation}
    F_\lambda=\frac{c}{\lambda^2}F_\nu
    \label{eq: conversion}
\end{equation}

Different species emit light in distinct spectral regions. To classify these regions, the entire spectrum was split into windows of 0.01 $\mu$m from 4.9 to 27.5 $\mu$m. The window size of 0.01 $\mu$m was chosen, as it is approximately the expected window size at this spectral resolution ($R=\lambda/\Delta\lambda\Rightarrow\Delta\lambda=\frac{\lambda}{R}\approx\frac{30 \mu \mathrm{m}}{3000}=0.01 \mu \mathrm{m}$). Next, we integrate the emission for each species in that window and note the species that has the highest flux in that region. Doing this for all models and taking the mode gives the species for each window that has the strongest emission for most of the simulated spectra.  


\section{Cross-Correlation Technique}
For our analysis, we used cross-correlation. This is a signal detection technique used to detect weak signals. This technique was implemented to detect the weak spectral features of NO and NH\3. It is defined as follows: 

\begin{equation}
    R_{fg}(\tau)=\int^\infty_{-\infty}\overline{f(t)}g(t+\tau)dt
\end{equation}

where $R_{fg}(\tau)$ is the cross-correlation, $f$ and $g$ are functions of $t$, and $\tau$ is the lag. As all our functions are real valued, we can omit the complex conjugation in the integrand.

\begin{figure}[H]
    \centering
    \includegraphics[width=.5\linewidth]{VISUALIZE CROSSCORRELATION.pdf}
    \caption{Caption}
    \label{fig: crosscorr vis}
\end{figure}

\autoref{fig: crosscorr} illustrates how the cross correlation can be used to detect a weak signal and its location. On the left, the added Gaussian is barely visible, whereas on the right, there is a clear peak, which signals that the weak signal is present in the noise and its location.

To apply this technique to spectra to test for the presence of specific molecules, we used the emissions for each species across the model grid. The emission per molecule for each model was added together and divided by the maximum to normalize them. This template emission was cross-correlated with the spectrum containing emission from all the species. The cross-correlation should peak when the template spectrum matches the spectrum containing all the species. To quantify the height of the peak, we calculated the difference between the height of the peak and the median value of the cross-correlation in a small region around the peak. The median, which was selected in favor of the mean as it is less susceptible to outliers, captures the general trend around the peak. Hypothesis testing was implemented to test whether or not there is statistical evidence to say the molecule is present in the spectrum. We had the following null hypothesis and alternative hypothesis, with the difference between the height of the peak and the median of neighboring points as the test statistic. 

\begin{equation}
    H_0: \text{The height of the peak in the cross-correlation is caused by noise}
\end{equation}

\begin{equation}
    H_1: \text{The height of the peak in the cross-correlation is caused by the molecule}
\end{equation}

To assess the significance of the peak in cross-correlation, a Monte Carlo simulation was used to simulate 10000 synthetic spectra under the null hypothesis. First, all the flux values for different wavelength values are collected in a set. Second, flux values were sampled with replacement from the set until a collection of values was obtained that was the same size as the original set. This procedure is akin to bootstrap sampling. In this way, a completely randomized spectrum is created, which simulates pure noise. Third, we calculated the cross-correlation of the randomized spectrum with the template and determined the difference between the peak height and the median value of the cross-correlation in a small region around the peak. Repeating these steps to create 10000 spectra, and with it corresponding values for the test statistic, resulted in a distribution. The peak is deemed significant, and thus the molecule is present, when its test statistic exceeds 95\% of the test statistic values from the randomized spectra ($\alpha=0.05$).


\section{Observations}
After investigating the simulated spectra, the techniques were extended to observations made with JWST MIRI MRS. We have access to 3 sources: GWLup \citep{Gasman_2023}, Sz98 \citep{Grant_2023}, and V1094Sco \citep{taboneinprepp}. The spectra we received were already continuum-subtracted and are shown in \autoref{fig: Measurements}. However, the continuum subtraction left artifacts in the spectrum. \cite{Grant_2023} has listed the regions in the spectrum containing such artifacts, and that list of regions was used to mask out the artefacts. 

\begin{figure}[H]
    \centering
    \includegraphics[width=\linewidth]{Figures/Measurements.pdf}
    \caption{The continuum subtracted JWST MIRI MRS spectra of GWLup \citep{Gasman_2023}, Sz98 \citep{Grant_2023}, and V1094Sco \citep{taboneinprepp}. Some of the notable spectral features are labeled using insets.}
    \label{fig: Measurements}
\end{figure}

Furthermore, LTE models were used to fit the observed spectra. A similar procedure to the process described by \cite{Grant_2023} was followed. 

\begin{equation}
    \chi^2=\frac{1}{N}\sum_{i=1}^N\frac{(F_{i,obs}-F_{i, mod})^2}{\sigma^2}
\end{equation}

where N is the number of spectral elements and $\sigma^2$ the variance in an emission-free region of the spectrum, the degrees of freedom (DOF) of the fit are two: column-density and temperature. To get the accurate spectrum, the distance and radial velocity (\autoref{tab: dist+rv}) must be taken into account. The LTE model spectrum needs to be multiplied by the emitting area, which is assumed to be a disk with radius $R$. However, as this is just a scaling factor $R$ is not considered a DOF. The confidence intervals corresponding to 1$\sigma$, 2$\sigma$, and 3$\sigma$ are given by $\chi^2_{min}+2.3$, $\chi^2_{min}+6.2$, and $\chi^2_{min}+11.8$ respectively. The emission of the molecules was fitted iteratively. First, the emission per molecule was fitted to the spectrum. Second, the fits were improved by subtracting the fitted emission of all other molecules from the first step, except for the one that was being fitted, and fitting to the residuals. Lastly, the previous step is repeated one more time. 
The fitted spectra were used to find the emission of specific molecules and subtract them from the spectrum. This was used to find weaker emissions of other molecules in the residuals. 

\begin{table}[H]
\centering
\begin{tabular}{ccc}
\hline
Source   & \makecell{Distance \\\citep{henning2024mindsjwstmirimidinfrared}}  & \makecell{Radial velocity \\ \citep{Frasca_2017}} \\ \hline
GWLup    & 155.20 pc & -3.3 km/s       \\
Sz98     & 156.27 pc & -1.4 km/s       \\
V1094Sco & 152.44 pc & 2.2 km/s        \\ \hline
\end{tabular}
\caption{The distances and radial velocities of GWLup, Sz98, and V1094Sco}
\label{tab: dist+rv}
\end{table}

To produce an estimate on the upper limit, we used Bayes' theorem to find the posterior distribution of the parameters. 
\begin{equation}
    P(model|data)\propto P(data|model)P(model)
    \label{eq: Bayes}
\end{equation}
where $P(model|data)$ is the posterior distribution, $P(data|model)$ the likelihood, and $P(model)$ the prior. Assuming that the residuals are normally distributed gives the likelihood
\begin{equation}
    P(data|model)\propto\exp\left(-\frac{\chi^2}{2}\right)
\end{equation}
The ProDiMo models were used to set a lower limit on the temperature, which is implemented in the flat prior
\begin{equation}
    P(model) = 
    \begin{cases}
        1, & \text{if } T > T_0 \\
        0, & \text{if } T \leq T_0
    \end{cases}
\end{equation}
We combined the likelihood and prior using \autoref{eq: Bayes} to get the posterior distribution. Next, samples were drawn from this distribution. The upper limit on the column density was derived from the samples as the value below which 95\% of the samples fall. 


\chapter{Results}\label{Ch: Results}
\section{ProDiMo Output}

\begin{figure}[H]
    \centering
    \includegraphics[width=\linewidth]{Figures/DensityTemperature.pdf}
    \caption{The density and temperature maps of gas and dust in the disk of the fiducial model. The horizontal axis shows the distance $r$ from the host star, and the vertical axis shows the height above the midplane $z$ divided by the radius $r$.}
    \label{fig: tempdensity}
\end{figure}

% \textbf{COMPARING NON-ENHANCED TO ENHANCED}
After running the models, we analyzed their output. First, we looked at the densities of the gas and dust in the disk. The densities of the fiducial model are shown in the top row of \autoref{fig: tempdensity}. The density of the gas is more vertically extended than the density of the dust. This is expected, as the dust settles around the midplane. Furthermore, the gas density stretches farther from the host star than the dust. 

Next, we inspected the temperatures of gas and dust across the disk. Those are shown in the bottom row of \autoref{fig: tempdensity}.  The gas and dust temperatures vary across the disk for different heights and radii. However, they are equal in the midplane as they are coupled. Above the midplane, they decouple, and the temperatures of the gas and dust change. The gas temperatures are higher as they get heated by the star, whereas the dust temperatures are lower. Notably, the temperature of the dust is vertically isothermal. This result is expected as the density of the dust is orders of magnitude lower than that of the gas in the upper regions of the disk.  As a result, starlight passes through with minimal obstruction, leading to little variation in temperature with height.

% \begin{figure}[H]
%     \centering
%     \includegraphics[width=\linewidth]{Figures/Temperature.pdf}
%     \caption{The temperature profile of gas and dust in the disk of the fiducial model.}
%     \label{fig: temperature}
% \end{figure}


\begin{figure}[H]
    \centering
    \includegraphics[width=\linewidth]{Figures/Abundance1.pdf}
    \caption{The distribution of H\2O, CO\2, OH, and CO across the disk of the fiducial model. The line origins of molecules in the wavelength range of JWST MIRI MRS are highlighted in cyan. The horizontal axis shows the distance $r$ from the host star, and the vertical axis shows the height above the midplane $z$ divided by the radius $r$.}
    \label{fig: others distribution}
\end{figure}
\lipsum[1]
\begin{figure}[H]
    \centering
    \includegraphics[width=.8\linewidth]{Figures/AbundanceN2.pdf}
    \caption{The distribution of N\2 across the disk of the fiducial model. The line origin of N\2 in the wavelength range of JWST MIRI MRS is highlighted in cyan. The horizontal axis shows the distance $r$ from the host star, and the vertical axis shows the height above the midplane $z$ divided by the radius $r$.}
    \label{fig: n2 distribution}
\end{figure}
\lipsum[2]
\lipsum[2]
\begin{figure}[H]
    \centering
    \includegraphics[width=\linewidth]{Figures/Abundance2.pdf}
    \caption{The distribution of HCN, HNC, NH\3, and NO across the disk of the fiducial model. The line origins of molecules in the wavelength range of JWST MIRI MRS are highlighted in cyan. The horizontal axis shows the distance $r$ from the host star, and the vertical axis shows the height above the midplane $z$ divided by the radius $r$.}
    \label{fig: nitrogen distribution}
\end{figure}
Lastly, we inspected the distribution of nitrogen-bearing molecules: N\2 (\autoref{fig: n2 distribution}), HCN, HNC, NH\3, and NO (\autoref{fig: nitrogen distribution}). The density of the molecules changes a lot for different radii and heights. For example, in the region between 0.05 and 0.1 au, HCN decreases compared to the surrounding areas, but HNC increases in the same region. There is a gap in the densities between 1 and 200 au. In these areas, the temperature drops below the freezing point of these molecules, and they are well shielded from UV radiation, resulting in the formation of ice. 
\newpage
\section{FLiTs Spectra}
Consistent mid-infrared spectra were calculated using FLiTs for all the models in the grid. The resulting spectra are shown in \autoref{fig: all spectra}. There is a clear division in the shapes of the spectra. In the bottom left corner, 6 spectra have a strong C\2H\2 feature. These models also have a C/O ratio greater than unity. The other models have strong water features and have a C/O ratio smaller than unity. 

\begin{figure}[H]
    \centering
    \includegraphics[width=\linewidth]{Figures/All_spectra.pdf}
    \caption{The simulated spectra of all the models in the model grid using FLiTs. On the horizontal axis, the O abundance is varied from -0.5 to 0.5 w.r.t. the reference abundance. The C abundance is varied with -0.5 to 0.5 w.r.t. the reference abundance from top to bottom. Some notable spectral features are labeled.}
    \label{fig: all spectra}
\end{figure}
\subsection{Total Flux across Grid}
The flux of the different species changes depending on the abundances of C and O. \autoref{fig: Heatmaps1} shows the total flux for CO, CO\2, H\2O, and OH across the grid of models. The total flux for C\2H\2, HCN, NO, and NH\3 for all the models in the grid is shown in \autoref{fig: Heatmaps2}. 

\begin{figure}[H]
    \centering
    \includegraphics[width=\linewidth]{Figures/Heatmaps1.pdf}
    \caption{The total flux of CO, CO\2, H\2O, and OH for all the models in the grid.}
    \label{fig: Heatmaps1}
\end{figure}

\lipsum[1-2]
\begin{figure}[H]
    \centering
    \includegraphics[width=\linewidth]{Figures/Heatmaps2.pdf}
    \caption{The total flux of C\2H\2, HCN, NO, and NH\3 for all the models in the grid.}
    \label{fig: Heatmaps2}
\end{figure}

The HCN flux increases as C/O gets larger, and NO decreases. The flux of NH\3 generally increases as the C abundance gets lower. We hypothesize that the cause for this is the reactions that take place to form NH\3.
OH is an important component in the formation of NH\3. \textbf{CHECK CHEMISTRY}

\ce{NH2+ + OH -> NH3 + O+}

\textbf{MORE REACTIONS}\\
Furthermore, there could be some competition with HCN, where more N is locked in HCN when the C abundance is larger. 

\subsection{Regions of Interest}
Next, we wanted to identify regions in the spectrum that could help us detect NO and NH\3. We split the model grid into 2 groups. One group has C/O smaller than unity, and the other greater. We chose this distinction as the behaviors of the spectra are different. This difference is also visible in \autoref{fig: all spectra}. 

% \begin{figure}[H]
%     \centering
%     \includegraphics[width=\linewidth]{Figures/ClassificationCOgt0.pdf}
%     \caption{The regions of the spectrum where NO and NH\3 emission are the  strongest compared to other molecular emission in the models that have a C/O greater than unity}
%     \label{fig: class>1}
% \end{figure}

% \begin{figure}[H]
%     \centering
%     \includegraphics[width=\linewidth]{Figures/ClassificationCOst0.pdf}
%     \caption{The regions of the spectrum where NO and NH\3 emission are strongest in the models that have a C/O smaller than unity}
%     \label{fig: class<1}
% \end{figure}

\begin{figure}[H]
    \centering
    \includegraphics[width=\linewidth]{Figures/ClassificationCO.pdf}
    \caption{The regions of the spectrum where NO and NH\3 emission are strongest in the models for models that have a C/O greater than unity (top) and smaller than unity (bottom)}
    \label{fig: classes}
\end{figure}

In the top row of \autoref{fig: classes}, it is visible that the region between 4.9$\mu$m and 6 $\mu$m has many spectral regions where NO emission is strongest. NH\3 has the brightest emission compared to all the other molecules between 6 $\mu$m and 6.4 $\mu$m. Between 10 $\mu$m and 12 $\mu$m there are some smaller regions where NH\3 is strongest. In contrast, in the bottom row of \autoref{fig: classes}, the spectral regions where NO or NH\3 emission is strongest are more barren. The NO emission regions are not the strongest anymore, and neither is the NH\3 emission region next to it. The most promising region for NH\3 is now between 10 $\mu$m and 12 $\mu$m. 

\begin{figure}[H]
    \centering
    \includegraphics[width=\linewidth]{Figures/NO_region.pdf}
    \caption{A zoomed-in region of the spectrum of the fiducial model between 4.9 $\mu$m and 6 $\mu$m where NO emission is strongest.}
    \label{fig: NO region}
\end{figure}

\lipsum[2]

\begin{figure}[H]
    \centering
    \includegraphics[width=\linewidth]{Figures/NH3_region1.pdf}
    \caption{A zoomed-in region of the spectrum of the model with C+0.25 and O-0.25 between 6 $\mu$m and 6.5 $\mu$m where NH\3 emission is strongest.}
    \label{fig: NH3 region 1}
\end{figure}

\lipsum[2]

\begin{figure}[H]
    \centering
    \includegraphics[width=\linewidth]{Figures/NH3_region2.pdf}
    \caption{A zoomed-in region of the spectrum of the model with C+0.25 and O-0.25 between 10 $\mu$m and 12 $\mu$m where NO emission is strongest.}
    \label{fig: NH3 region 2}
\end{figure}

\lipsum[2]

\begin{figure}[H]
    \centering
    \includegraphics[width=\linewidth]{Figures/NH3_region3.pdf}
    \caption{A zoomed-in region of the spectrum of the fiducial model between 10 $\mu$m and 12 $\mu$m where NO emission is strongest.}
    \label{fig: NH3 region 3}
\end{figure}

The spectra in \autoref{fig: NO region}, \autoref{fig: NH3 region 1}, \autoref{fig: NH3 region 2}, and \autoref{fig: NH3 region 3} are all without any noise. When we added noise with SNR=300 to the same spectrum shown in figure \autoref{fig: NH3 region 1}, based on the methods described, we got the spectrum shown in figure \autoref{fig: add noise}. In this figure, the emission of NH\3 is obscured by noise at this SNR, making it much more difficult to detect.  

\begin{figure}[H]
    \centering
    \includegraphics[width=\linewidth]{Figures/AddNoise.pdf}
    \caption{A zoomed-in region of the spectrum of the model with C+0.25 and O-0.25 between 6 $\mu$m and 6.5 $\mu$m where NH\3 emission is strongest. A Gaussian noise corresponding to an SNR of 300 was added.}
    \label{fig: add noise}
\end{figure}

\section{Molecule Detection}
As visible in \autoref{fig: add noise}, it is hard to distinguish the NH\3 and NO from the spectrum visually. However, cross-correlation is a technique that can be used to find such weak signals. In calculating the cross-correlation, we used the average spectrum for each species of all the models in the model grid and normalized them. This template was then used to cross-correlate with the full spectrum. The cross-correlation of the H\2O template with the spectrum of the fiducial model is shown in \autoref{fig: crosscorr}. The peak signals that H\2O is present in the spectrum, and it indeed is present. 

\begin{figure}[H]
    \centering
    \includegraphics[width=.6\linewidth]{Figures/Cross-Correlation.pdf}
    \caption{The cross-correlation of the H\2O template with the spectrum the fiducial model}
    \label{fig: crosscorr}
\end{figure}

When we applied this method to the fiducial model and model C+0.25 O-0.25, we got the detections shown in \autoref{tab: combined_detections}. 

% \begin{table}[!ht]
% \centering
% \begin{tabular}{lll}
% \hline
% \textbf{Molecule} & \textbf{Spectrum} & \textbf{Spectrum - Target Flux} \\ \hline
% C\2H\2            & Non-detection     & Non-detection                   \\
% CH\4             & Non-detection     & Non-detection                   \\
% CO              & Detection         & Non-detection                   \\
% CO\2             & Detection         & Non-detection                   \\
% H\2O             & Detection         & Non-detection                   \\
% HCN             & Non-detection     & Non-detection                   \\
% NH\3             & Non-detection     & Non-detection                   \\
% NO              & Non-detection     & Non-detection                   \\
% OH              & Detection         & Non-detection                   \\ \hline
% \end{tabular}

% \label{tab: fiducial detection}
% \end{table}

% \begin{table}[!ht]
% \centering
% \begin{tabular}{lll}
% \hline
% \textbf{Molecule} & \textbf{Spectrum} & \textbf{Spectrum - Target Flux} \\ \hline
% C\2H\2            & Detection         & Non-detection                   \\
% CH\4             & Non-detection     & Non-detection                   \\
% CO              & Detection         & Non-detection                   \\
% CO\2             & Non-detection     & Non-detection                   \\
% H\2O             & Non-detection     & Non-detection                   \\
% HCN             & Detection         & Non-detection                   \\
% NH\3             & Non-detection     & Non-detection                   \\
% NO              & Non-detection     & Non-detection                   \\
% OH              & Non-detection     & Non-detection                   \\ \hline
% \end{tabular}

% \label{tab: other detection}
% \end{table}

\begin{table}[!ht]
\centering
\resizebox{\textwidth}{!}{
\begin{tabular}{|l|ccc|ccc|}
\hline
\textbf{Molecule} & \textbf{A: Detected} & \textbf{A: Residual} & \textbf{A: Confirmed} & \textbf{B: Detected} & \textbf{B: Residual} & \textbf{B: Confirmed} \\
\hline
C$_2$H$_2$ & No  & No  & --           & Yes & No  & \ding{51} \\
CH$_4$     & No  & No  & --           & No  & No  & --        \\
CO         & Yes & No  & \ding{51}    & Yes & No  & \ding{51} \\
CO$_2$     & Yes & No  & \ding{51}    & No  & No  & --        \\
H$_2$O     & Yes & No  & \ding{51}    & No  & No  & --        \\
HCN        & No  & No  & --           & Yes & No  & \ding{51} \\
NH$_3$     & No  & No  & --           & No  & No  & --        \\
NO         & No  & No  & \ding{51}    & No  & No  & --        \\
OH         & Yes & No  & \ding{51}    & No  & No  & --        \\
\hline
\end{tabular}
}
\caption{Molecular detections in the spectrum of the fiducial model (A) and the model with C+0.25 and O-0.25 (B) before and after subtracting molecular emission. A confirmed detection (\ding{51}) indicates that the molecule is detected in the full spectrum. A molecular emission is considered present in the full spectrum when the integrated flux over the wavelength range is greater than the integrated flux of a flat continuum at 0.1 mJy over the wavelength range.}
\label{tab: combined_detections}
\end{table}

We compared the cross-correlation technique for all the molecules. We did this for the spectrum of the fiducial model. Once, we cross-correlate with the spectrum to see if the molecule is present, and once with the molecule emission subtracted from the spectrum to check that this does not result in a false-positive.


% As some of the molecules have limited ranges in which they emit, we adjusted the ranges in which we cross-correlate the template and the spectrum. The ranges are listed in table \ref{}

% \begin{tabular}{lcc}
% \hline
% \textbf{Species} & \textbf{Min ($\mu$m)} & \textbf{Max ($\mu$m)} \\
% \hline
% C2H2   & 7.1 & 15.6 \\
% CH4    & 6.3 & 9.7 \\
% CO     & 4.9 & 5.6 \\
% CO2    & 13.0 & 17.3 \\
% H2O    & 4.9 & 27.5 \\
% HCN    & 6.4 & 17.0 \\
% NH3    & 5.1 & 27.5 \\
% NO     & 4.9 & 6.6 \\
% OH     & 8.3 & 27.5 \\
% \hline
% \end{tabular}

As some of the flux of a species overlaps with the flux emitted by other species, it would make detection easier if the emissions of the other species were removed. This was easily done with the simulated data, as the simulation produced the spectra of the individual species. To test this, the cross-correlation technique was applied to only the emission of NO and NH\3 with the associated noise added. Repeating this 100 times with different random noise for every model in the model grid gave a detection rate of 100\% and 23.8\% for NO and NH\3, respectively.  Especially the region where NO is emitted is of interest, as the only other species that emit in that range are CO and H\2O. Removing the flux of CO and H\2O for the fiducial gave the spectrum shown in \autoref{fig: H2O and CO removed}.

\begin{figure}[H]
    \centering
    \includegraphics[width=\linewidth]{Figures/H2O_CO_removed.pdf}
    \caption{A zoomed-in region of the spectrum of the fiducial model between 4.9 $\mu$m and 6 $\mu$m. The emission of both H\2O and CO has been removed.}
    \label{fig: H2O and CO removed}
\end{figure}

\section{Application of Detection Methods to JWST Observations}
After the confirmation that this technique is valid on the simulated spectra, we applied it to real data to see if they are effective here as well.

\subsection{Molecule Detection}
Using the templates generated from the emission of the species for all the ProDiMo models in the grid, we cross-correlated the templates with the measured spectra. The results of this are shown in \autoref{tab: realdata}.

\begin{table}[!ht]
\centering
\begin{tabular}{|l|ccc|}
\hline
\textbf{Molecule} & \textbf{GWLup} & \textbf{Sz98} & \textbf{V1094Sco} \\ \hline
C2H2            & Detection      & Non-detection & Detection         \\
CH4             & Non-detection  & Non-detection & Non-detection     \\
CO              & Detection      & Detection     & Detection         \\
CO2             & Detection      & Detection     & Detection         \\
H2O             & Detection      & Detection     & Detection         \\
HCN             & Detection      & Detection     & Detection         \\
NH3             & Non-detection  & Non-detection & Non-detection     \\
NO              & Non-detection  & Non-detection & Non-detection     \\
OH              & Detection      & Detection     & Detection         \\ \hline
\end{tabular}

\caption{The detection of different molecules in the spectra of GWLup, Sz98, and V1094Sco using cross-correlation.}
\label{tab: realdata}
\end{table}


LTE models were then used to fit CO and H\2O in the region 4.9-6.5 $\mu$m to the measured spectra. This region was selected as it mostly contains emission of just H\2O, CO, and NO. Furthermore, it was previously shown in the simulated spectra that NO was detectable when removing all other emissions from the spectrum. The resulting fits are shown in \autoref{fig: fits}.

\begin{figure}[H]
    \centering
    \includegraphics[width=\linewidth]{Figures/Fits.pdf}
    \caption{The fitted H\2O and CO spectra using a LTE model between 4.9 $\mu$m and 6.5$\mu$m for GWLup, Sz98, and V1094Sco}
    \label{fig: fits}
\end{figure}

The fitted CO and H\2O emission was subsequently removed, and the cross-correlation technique was applied to the residuals of the spectrum. In GWLup and Sz98, no NO was detected. However, V1094Sco has a detection of NO. The residuals of V1094Sco and the scaled NO template are shown in \autoref{fig: no detect}. 
\begin{figure}[H]
    \centering
    \includegraphics[width=\linewidth]{Figures/NO_Detect.pdf}
    \caption{The fitted H\2O and CO spectra using a LTE model between 4.9 $\mu$m and 6.5$\mu$m for V1094Sco}
    \label{fig: no detect}
\end{figure}


\subsection{Upperlimits on NH3 and NO}
After subtracting the CO and H\2O emission from the spectra in the region between 4.9 $\mu$m and 6.5 $\mu$m, the only emission remaining is that of NO and NH\3. This was used to calculate the $\chi^2$ for different values of column density and temperature. This value can then be used to find the likelihood distribution. To update the prior, the temperature of the emitting regions of NO and NH\3 in the fiducial model was used (\autoref{fig: NONH3temp}).
\begin{figure}[H]
    \centering
    \includegraphics[width=0.5\linewidth]{}
    \caption{Caption}
    \label{fig: NONH3temp}
\end{figure}
Using this, we concluded that the minimum temperature of NO is 400 K and of NH\3 750 K. These values were filled into the prior
\begin{equation}
    P(model) = 
    \begin{cases}
        1, & \text{if } T > T_0 \\
        0, & \text{if } T \leq T_0
    \end{cases}
\end{equation}
where $T_o=400$ K for NO and $T_o=750$ K for NH\3. Multiplying the prior with the likelihood distribution gave the posterior distribution up to a normalization constant, which was found be integrating the posterior distribution over all values of $N$ and $T$ and setting that equal to 1. 10000 samples were drawn from the posterior distribution and the number density values below which 95\% of the values fall are shown in \autoref{tab: upper limits}.

\begin{table}[H]
\begin{tabular}{lll}
\hline
Source   & NO ($\log_{10}(\mathrm{N[cm^{-2}]}))$   & NH\3 ($\log_{10}(\mathrm{N[cm^{-2}]}))$\\ \hline
GWLup    & 16.5 & 21.3               \\
Sz98     & 17   & 20.2               \\
V1094Sco & 16.7 & 21.6               \\ \hline
\end{tabular}
\caption{The upper limits on the column density of NO and NH\3 for GWLup, Sz98, and V1094Sco}
\label{tab: upper limits}
\end{table}



\chapter{Discussion}\label{Ch: Discussion}
\section{Interpretation}
This thesis provides new insights into the detection of nitrogen carriers NO and NH\3 in the planet-forming regions of protoplanetary disks. The influence of C and O abundances on the spectra was analyzed, which gave interesting results. The spectra formed 2 groups: spectra dominated by H\2O emission for the models that had a C/O ratio smaller than unity, and spectra dominated by C\2H\2 emission for models with C/O greater than unity. Following this, emission of both NO and NH\3 was less obscured by water lines in the spectra of the carbon-rich models, which leads to the conclusion that carbon-rich sources are the most prominent sources for the detection of NO and NH\3 in the planet-forming regions of protoplanetary disks. 

However, a direct NO and NH\3 detection from the spectrum proves to be difficult due to the low intensity of the emission of these molecules and the level of noise. To combat this, we have developed a method using cross-correlation to detect weak spectral features. Using this technique on the simulated spectra generated using FLiTs gave the expected results. The most prominent features were detected in the spectrum, but not the weak emission from NO and NH\3. NO emission was detected in the spectrum of all the models after removing the emission of all other molecules CO and H\2O. However, only in 23.8\% of the spectra was the emission of NH\3 detected.


\section{Comparison to Other Works}
\cite{Grant_2023} have detected CO, CO\2, H\2O, HCN, C\2H\2 and OH in the spectrum of GWLup. These are the same molecules we detected using our cross-technique. In the spectrum of Sz98 \cite{Gasman_2023}, they have detected CO, H\2O, OH, CO\2, and HCN. These are again the same molecules we detected. The fact that we have the same results as they is a promising sign for the validity of our technique. 

\cite{groningenthesis} performed research similar to our own. However, they investigated the detection of SO\2 to learn more about the sulfur depletion problem. \textbf{MORE TEXT COMPARING OUR RESULTS}

\section{Limitations}
For the cross-correlation technique, we assume that the peak at lag=0 is purely from the molecule that is being tested. However, it could be the case that the cross-correlation of the noise with the template results in a peak, because the noise coincidentally follows a similar structure to the emission of the species. We suspect that this is highly unlikely, as the noise would then need to be in the same location (i.e., not shifted to different wavelengths) and follow a similar pattern (i.e., be bright in the same locations as the molecular emission). A problem that might be more impactful is fringing. Fringing is the effect when taking a spectrum that results in a regular pattern. This pattern could coincide with the pattern formed by the ro-vibrational transitions, which also has a regular spacing. If it were the case that these two phenomena would overlap, that could result in a false-positive.

Other than noise, other species could also interfere with our method of cross-correlation. When the emission of different spectra overlaps, that could result in a detection. For example, in \autoref{fig: NO region} the CO emission partly overlaps with the emission coming from NO. This, with the fact that the spacing of the two species is similar, could result in false positives. 

Following this, we use bootstrapping to create the hypothesis test by sampling the distribution of the test statistic. However, this destroys all patterns in the data. This can give a disproportionate view of the significance of a species' presence, leading to a false-positive.

Averaging the emission of all species across the entire grid may not be the best technique for creating a template. The different abundances heavily impact the shape of the emission, and combining them into a single template results in a spectrum that has about the same shape, but can lose 

\section{Future Work}
In this work, the only things that were varied were the C and O abundances. Some other properties could be interesting to investigate, for example, different disk structures. The distribution of NH\3 in \autoref{fig: nitrogen distribution}, is close to the midplane in the disk, resulting in it being hidden inside the disk. Different disk structures, such as gaps, could expose the NH\3 reservoir, allowing for detection. 

Furthermore, the stellar properties could affect the nitrogen carriers in the disk. Changing attributes such as stellar luminosity, mass, and temperature can result in different chemistries in the disk, thereby altering the spectrum that is measured. 

Improving models to account for additional species or more realistic noise.

Refining spectral resolution and detection limits using advanced filters.

Looking at VLMS paper aditya \citep{Arabhavi_2024}

\chapter{Conclusion}\label{Ch: Conclusion}
In this thesis, we have looked at the influence of the carbon and oxygen abundances on the nitrogen carriers in a protoplanetary disk and applied that knowledge to JWST MIRI MRS observations. Our main conclusions are as follows:
\begin{itemize}
    \item \textbf{SOMETHING ABOUT THE EFFECT OF C AND O ON FLUX}
    \item We have selected spectral regions in which the emission from NO and NH\3 is the strongest. In particular, sources with a C/O ratio greater than unity have the greatest potential for detecting NO and NH\3 as their emission is not obscured by H\2O emission.
    \item We developed a new molecule detection method using cross-correlation. Using the ProDiMo models in the grid, emission templates were created for each species. 
    \item We reconfirmed the known species present in the spectra of GWLup, Sz98, and V1094Sco using the cross-correlation technique.
    \item We made a possible detection of NO in the spectrum of V1094Sco.
    \item We determined upper limits on the column densities of NO and NH\3 for GWLup, Sz98, and V1094Sco.
\end{itemize}
\textbf{SOMETHING ABOUT THE FUTURE}
\bibliographystyle{aa.bst}
\bibliography{references}
\appendix
\chapter{More ProDiMo Results}
\chapter{Chi square fits}
\begin{figure}[!ht]
    \centering
    \begin{subfigure}[b]{0.49\textwidth}
        \centering
        \includegraphics[width=\textwidth]{radexpy_niels/Radexpy_for_Niels/chi2_map_H2O_GWLup.pdf}
        \caption{caption}
    \end{subfigure}
    \hfill
    \begin{subfigure}[b]{0.49\textwidth}
        \centering
        \includegraphics[width=\textwidth]{radexpy_niels/Radexpy_for_Niels/chi2_map_CO_GWLup.pdf}
        \caption{caption}
    \end{subfigure}
    \caption{caption}
\end{figure}

\begin{figure}[!ht]
    \centering
    \begin{subfigure}[b]{0.49\textwidth}
        \centering
        \includegraphics[width=\textwidth]{radexpy_niels/Radexpy_for_Niels/chi2_map_H2O_Sz98.pdf}
        \caption{caption}
    \end{subfigure}
    \hfill
    \begin{subfigure}[b]{0.49\textwidth}
        \centering
        \includegraphics[width=\textwidth]{radexpy_niels/Radexpy_for_Niels/chi2_map_CO_Sz98.pdf}
        \caption{caption}
    \end{subfigure}
    \caption{caption}
\end{figure}

\begin{figure}[!ht]
    \centering
    \begin{subfigure}[b]{0.49\textwidth}
        \centering
        \includegraphics[width=\textwidth]{radexpy_niels/Radexpy_for_Niels/chi2_map_H2O_V1094Sco.pdf}
        \caption{caption}
    \end{subfigure}
    \hfill
    \begin{subfigure}[b]{0.49\textwidth}
        \centering
        \includegraphics[width=\textwidth]{radexpy_niels/Radexpy_for_Niels/chi2_map_CO_V1094Sco.pdf}
        \caption{caption}
    \end{subfigure}
    \caption{caption}
\end{figure}
\end{document}